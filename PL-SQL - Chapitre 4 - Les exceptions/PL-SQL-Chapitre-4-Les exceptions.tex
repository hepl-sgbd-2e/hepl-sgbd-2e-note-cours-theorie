\documentclass[10pt]{beamer}

\input{/Users/daniel/Documents/LaTeX/beamer-style.tex}

\title{SGBD - 2\textsuperscript{e}}
\subtitle{PL-SQL - Chapitre 4 - Les exceptions}
\date{\today}
\author{Daniel Schreurs}
\institute{Haute École de Province de Liège}
%\titlegraphic{\hfill\includegraphics[height=1.5cm]{logo.eps}}

\setbeamertemplate{frame footer}{\insertsectionhead}
\begin{document}
\maketitle

\setbeamerfont{subsection in toc}{size=\small}
\setbeamerfont{subsubsection in toc}{size=\normalsize}
\setbeamertemplate{section in toc}[sections numbered]
\setbeamertemplate{subsection in toc}[subsections numbered]
\setbeamertemplate{subsubsection in toc}[subsubsections numbered]
\begin{frame}[allowframebreaks]{Table des matières du chapitre}
    \tableofcontents[subsectionstyle=show/show/hide,subsubsectionstyle=show/show/hide,]
\end{frame}

\section{Utilisation des exceptions}
\tocss
\subsection{Déclaration des exceptions}
\begin{frame}{\secname : \subsecname}
    \begin{itemize}
        \item Dans la section \lstinline[language=sql]!DECLARE!
        \item \lstinline[language=sql]!Nom\_de\_l\_exception EXCEPTION;!
        \item Lorsqu'une erreur survient, elle est lancée dans le \emph{gestionnaire d'exception}.
        \item À nous d'en faire quelque chose.
    \end{itemize}
\end{frame}

\subsection{Le gestionnaire d'exceptions}
\begin{frame}{\secname : \subsecname}
    \begin{itemize}
        \item Est chargé d'intercepter les différentes erreurs
        \item Peut être déclaré dans n'importe quel bloc PL/SQL
        \item Commence par le mot clé \lstinline[language=sql]!EXCEPTION!
    \end{itemize}
    Une exception est lancée :
    \begin{itemize}
        \item Explicitement par un RAISE
        \item Implicitement par Oracle (exceptions prédéfinies)
    \end{itemize}
\end{frame}

\subsection{La clause WHEN OTHERS}
\begin{frame}{\secname : \subsecname}
    \begin{itemize}
        \item La clause \lstinline[language=sql]!WHEN OTHERS! dans un gestionnaire d'exceptions permet d'intercepter n'importe quel type d'exceptions non géré par ailleurs.
        \item La définition de cette clause devrait devenir systématique.
    \end{itemize}
\end{frame}

\begin{frame}{\secname : \subsecname}
    \lstinputlisting[language=sql, title=La clause WHEN OTHERS]{../exemples/PLSQL Chapitre 4/exception.sql}
\end{frame}

\section{Catégories d'exceptions}
\tocss

\begin{frame}{\secname}
    \begin{figure}
        \begin{center}
            \includegraphics[width=0.8\textwidth]{../assets/img/catégories.png}
            \caption{Catégories d'exceptions}
        \end{center}
    \end{figure}
\end{frame}

\subsection{Exceptions définies "non redirigées"}
\begin{frame}{\secname : \subsecname}
    \begin{itemize}
        \item Doivent être déclarées
        \item Doivent être lancées par le programmeur avec la commande RAISE
        \item Ne sont pas associées à une erreur prédéfinie par Oracle
    \end{itemize}
    \metroset{block=fill}
    \begin{alertblock}{Important}
        Ce type d'exception est utile pour gérer une exception spécifique liée à la logique de programmation du bloc PL/SQL.
    \end{alertblock}

\end{frame}

\begin{frame}{\secname : \subsecname}
    \lstinputlisting[language=sql, title=Exceptions définies "non redirigées"]{../exemples/PLSQL Chapitre 4/exception2.sql}
\end{frame}
\begin{frame}{\secname : \subsecname}
    \begin{table}[]
        \begin{tabular}{lll}
            Exception                    & Erreur ORACLE      & Valeur du SQLCODE \\
            ACCESS\_INTO\_NULL           & ORA-06530          & -6530             \\
            CASE\_NOT\_FOUND             & ORA-06592          & -6592             \\
            COLLECTION\_IS\_NULL         & ORA-06531          & -6531             \\
            CURSOR\_ALREADY\_OPEN        & ORA-06511          & -6511             \\
            \textbf{DUP\_VAL\_ON\_INDEX} & \textbf{ORA-00001} & \textbf{-1}       \\
            INVALID\_CURSOR              & ORA-01001          & -1001             \\
            INVALID\_NUMBER              & ORA-01722          & -1722             \\
            LOGIN\_DENIED                & ORA-01017          & -1017             \\
            \textbf{NO\_DATA\_FOUND}     & \textbf{ORA-01403} & \textbf{+100}     \\
            NOT\_LOGGED\_ON              & ORA-01012          & -1012
        \end{tabular}
    \end{table}
\end{frame}
\begin{frame}{\secname : \subsecname}
    % Please add the following required packages to your document preamble:
    % \usepackage[table,xcdraw]{xcolor}
    % If you use beamer only pass "xcolor=table" option, i.e. \documentclass[xcolor=table]{beamer}
    \begin{table}[]
        \begin{tabular}{lll}
            \multicolumn{1}{c}{{\color[HTML]{333333} Exception}} & \multicolumn{1}{c}{{\color[HTML]{333333} Erreur ORACLE}} & \multicolumn{1}{c}{{\color[HTML]{333333} Valeur du SQLCODE}} \\
            {\color[HTML]{333333} PROGRAM\_ERROR}                & {\color[HTML]{333333} ORA-06501}                         & {\color[HTML]{333333} -6501}                                 \\
            {\color[HTML]{333333} ROWTYPE\_MISMATCH}             & {\color[HTML]{333333} ORA-06504}                         & {\color[HTML]{333333} -6504}                                 \\
            {\color[HTML]{333333} SELF\_IS\_NULL}                & {\color[HTML]{333333} ORA-30625}                         & {\color[HTML]{333333} -30625}                                \\
            {\color[HTML]{333333} STORAGE\_ERROR}                & {\color[HTML]{333333} ORA-06500}                         & {\color[HTML]{333333} -6500}                                 \\
            {\color[HTML]{333333} SUBSCRIPT\_BEYOND\_COUNT}      & {\color[HTML]{333333} ORA-06533}                         & {\color[HTML]{333333} -6533}                                 \\
            {\color[HTML]{333333} SUBSCRIPT\_OUTSIDE\_LIMIT}     & {\color[HTML]{333333} ORA-06532}                         & {\color[HTML]{333333} -6532}                                 \\
            {\color[HTML]{333333} SYS\_INVALID\_ROWID}           & {\color[HTML]{333333} ORA-01410}                         & {\color[HTML]{333333} -1410}                                 \\
            {\color[HTML]{333333} TIMEOUT\_ON\_RESOURCE}         & {\color[HTML]{333333} ORA-00051}                         & {\color[HTML]{333333} -51}                                   \\
            {\color[HTML]{333333} \textbf{TOO\_MANY\_ROWS}}      & {\color[HTML]{333333} \textbf{ORA-01422}}                & {\color[HTML]{333333} \textbf{-1422}}                        \\
            {\color[HTML]{333333} \textbf{VALUE\_ERROR}}         & {\color[HTML]{333333} \textbf{ORA-06502}}                & {\color[HTML]{333333} \textbf{-6502}}                        \\
            {\color[HTML]{333333} ZERO\_DIVIDE}                  & {\color[HTML]{333333} ORA-01476}                         & {\color[HTML]{333333} -1476}
        \end{tabular}
    \end{table}
\end{frame}

\subsection{Exceptions prédéfinies : NO\_DATA\_FOUND}
\begin{frame}{\secname : \subsecname}
    \lstinputlisting[language=sql, title=Le nr de l'employé n’existe pas]{../exemples/PLSQL Chapitre 4/exception3.sql}
\end{frame}

\subsection{Exceptions prédéfinies : TOO\_MANY\_ROWS}
\begin{frame}{\secname : \subsecname}
    \lstinputlisting[language=sql, title=Plus d'un employé SCOTT]{../exemples/PLSQL Chapitre 4/exception4.sql}
\end{frame}

\subsection{Exceptions prédéfinies : DUP\_VAL\_ON\_INDEX}
\begin{frame}{\secname : \subsecname}
    \lstinputlisting[language=sql, title=Erreur de données]{../exemples/PLSQL Chapitre 4/exception5.sql}
\end{frame}

\subsection{Exceptions prédéfinies : INVALID\_NUMBER}
\begin{frame}{\secname : \subsecname}
    \lstinputlisting[language=sql, title=Erreur de données]{../exemples/PLSQL Chapitre 4/exception6.sql}
\end{frame}

\subsection{Exceptions prédéfinies : VALUE\_ERROR}
\begin{frame}{\secname : \subsecname}
    \lstinputlisting[language=sql, title=Chaine réceptrice trop petite !]{../exemples/PLSQL Chapitre 4/exception7.sql}
\end{frame}

\subsection{Exceptions définies "redirigées"}
\begin{frame}{\secname : \subsecname}
    \begin{itemize}
        \item Une exception définie par le programmeur peut être associée au code d'erreur d'une exception définie par Oracle
        \item Oracle se charge de lancer ce type d'exception : pas besoin de lancer un \lstinline[language=sql]!RAISE!
        \item Une exception redirigée se définit en utilisant le pragma \lstinline[language=sql]!EXCEPTION\_INIT!
    \end{itemize}
\end{frame}

\begin{frame}{\secname : \subsecname}
    Techniques des exceptions définies "redirigées" :
    \begin{itemize}
        \item Déclarer l'exception : \lstinline[language=sql]!NameExc EXCEPTION!
        \item Associer le nom de l'exception à un numéro d'erreur Oracle \lstinline[language=sql]!PRAGMA EXCEPTION_INIT (NameExc, Oracle_error);!
        \item Utiliser l'exception dans la section des gestions d'exceptions.
    \end{itemize}
    \metroset{block=fill}
    \begin{alertblock}{Important}
        Il ne faut pas lancer explicitement l'exception !
    \end{alertblock}
\end{frame}

\subsection{Exceptions définies "redirigées" : intégrité référentielle}
\begin{frame}{\secname : \subsecname}
    \lstinputlisting[language=sql, title=Contrainte d'intégrité sur Mgr est violée]{../exemples/PLSQL Chapitre 4/EXCEPTION-INIT.sql}
\end{frame}

\subsection{Exceptions définies "redirigées" : Ressource Busy}
\begin{frame}{\secname : \subsecname}
    \lstinputlisting[language=sql, title=La ressource est occupée]{../exemples/PLSQL Chapitre 4/EXCEPTION-INIT2.sql}
\end{frame}

\subsection{Exceptions définies "redirigées" : Ressource Busy}
\begin{frame}{\secname : \subsecname}
    \metroset{block=fill}
    \begin{alertblock}{Important}
        Les exceptions remontent de bloc en bloc jusqu'au code appelant si elles ne sont pas capturées.
    \end{alertblock}
\end{frame}

\begin{frame}{\secname : \subsecname}
    \lstinputlisting[language=sql, title=Une erreur non interceptée]{../exemples/PLSQL Chapitre 4/EXCEPTION-INIT3.sql}
\end{frame}

\begin{frame}{\secname : \subsecname}
    \lstinputlisting[language=sql, title=Une erreur interceptée]{../exemples/PLSQL Chapitre 4/EXCEPTION-INIT4.sql}
\end{frame}

\begin{frame}{\secname : \subsecname}
    \lstinputlisting[language=sql, title=Une erreur non interceptée dans un bloc imbriqué]{../exemples/PLSQL Chapitre 4/EXCEPTION-INIT5.sql}
\end{frame}

\begin{frame}{\secname : \subsecname}
    \lstinputlisting[language=sql, title=Une erreur interceptée dans un bloc imbriqué]{../exemples/PLSQL Chapitre 4/EXCEPTION-INIT6.sql}
\end{frame}

\begin{frame}{\secname : \subsecname}
    \lstinputlisting[language=sql, title=L'instruction NULL]{../exemples/PLSQL Chapitre 4/EXCEPTION-INIT7.sql}
\end{frame}

\section{Petit guide...}
\begin{frame}{\secname}
    \begin{itemize}
        \item Utiliser systématiquement la déclaration d'un gestionnaire d'exception dans tout bloc PL/SQL où une erreur est susceptible d'intervenir;
        \item Laisser remonter les erreurs quand vous ne savez pas les gérer;
        \item Profitez des exceptions pour gérer les transactions;
    \end{itemize}
\end{frame}

\begin{frame}{\secname}
    Les erreurs surviennent généralement :
    \begin{itemize}
        \item Pendant les opérations arithmétiques;
        \item Les opérations sur la base de données (\lstinline[language=sql]!SELECT!, \lstinline[language=sql]!INSERT!, \lstinline[language=sql]!UPDATE!, \lstinline[language=sql]!DELETE!);
        \item Ainsi que les conversions de données et/ou manipulations de chaînes de caractères.
    \end{itemize}
\end{frame}

\begin{frame}{\secname}
    \begin{itemize}
        \item Définir des exceptions si le code écrit reçoit des encodages de données erronées ou ne respectant pas un format ou un type prédéfini.  Exemple : les paramètres \lstinline[language=sql]!NULL! ou les sélections qui ne ramènent aucune donnée
              Utiliser les attributs \lstinline[language=sql]!\%TYPE! et \lstinline[language=sql]!\%ROWTYPE! pour rendre le code indépendant des données accédées et des modifications, ajouts ou suppressions de colonnes.
        \item Pour chaque exception interceptée, se poser la question de savoir si les opérations effectuées dans la base de données doivent être confirmées ou annulées (\lstinline[language=sql]!COMMIT! ou \lstinline[language=sql]!ROLLBACK! ?)
    \end{itemize}
\end{frame}

\begin{frame}{\secname}
    \metroset{block=fill}
    \begin{alertblock}{Important}
        Les exceptions peuvent faire partie intégrante de la logique de programmation.
    \end{alertblock}
    Dans le code d'une exception, un traitement particulier peut être effectué.
\end{frame}

\begin{frame}{\secname}
    \lstinputlisting[language=sql, title=Traitement dans le gestionnaire d'exceptions]{../exemples/PLSQL Chapitre 4/EXCEPTION-INIT8.sql}
\end{frame}

\end{document}