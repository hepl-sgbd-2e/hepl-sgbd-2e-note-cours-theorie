\documentclass[10pt]{beamer}

\input{/Users/daniel/Documents/LaTeX/beamer-style.tex}

\title{SGBD - 2\textsuperscript{e}}
\subtitle{Chapitre 8 - Contraintes d'intégrités et déclencheurs}
\date{\today}
\author{Daniel Schreurs}
\institute{Haute École de Province de Liège}
%\titlegraphic{\hfill\includegraphics[height=1.5cm]{logo.eps}}


\setbeamertemplate{frame footer}{\insertsectionhead}
\begin{document}
\maketitle

\setbeamerfont{subsection in toc}{size=\small}
\setbeamerfont{subsubsection in toc}{size=\normalsize}
\setbeamertemplate{section in toc}[sections numbered]
\setbeamertemplate{subsection in toc}[subsections numbered]
\setbeamertemplate{subsubsection in toc}[subsubsections numbered]
\begin{frame}[allowframebreaks]{Table des matières du chapitre}
    \tableofcontents[subsectionstyle=show/show/hide,subsubsectionstyle=show/show/hide,]
\end{frame}

\section{Introduction}
\tocss
\subsection{Définition}
\begin{frame}{\secname : \subsecname}
    Ce chapitre s'inscrit dans le cadre général de l'intégrité des données.  Il a pour objectif d'en décrire la partie contrôle sémantique.
    Le contrôle sémantique des données : assure la cohérence des informations stockées par rapport à leur signification dans la réalité
    Exemples :
    \begin{itemize}
        \item La cote d'un élève doit être comprise entre 0 et 20
        \item La valeur d'un stock ne peut être négative
    \end{itemize}
\end{frame}

\begin{frame}{\secname : \subsecname}
    Dans une base de données relationnelle :
    \begin{itemize}
        \item Contraintes inhérentes au modèle relationnel
        \item Contraintes liées à une application particulière
    \end{itemize}
\end{frame}

\subsection{Contraintes inhérentes au modèle relationnel}
\begin{frame}{\secname : \subsecname}
    Ou contraintes structurelles\footnote{(voir chapitre 2  : MRD et chapitre 3 : LDD)
    } :
    \begin{itemize}
        \item Intégrité de domaine
        \item Intégrité de relation
        \item Intégrité de référence
    \end{itemize}
\end{frame}

\subsection{Contraintes liées à une application particulière}
\begin{frame}{\secname : \subsecname}
    Ou \textbf{contraintes applicatives} ou contraintes spécifiques :\\
    Il s'agit d'une condition que doit vérifier un sous-ensemble de la base afin que l'on puisse affirmer que les informations sont cohérentes.
\end{frame}

\begin{frame}{\secname}
    \begin{itemize}
        \item Il faut pouvoir exprimer un ensemble de contraintes que doit satisfaire la base;
        \item Il faut que le SGBD puisse, au cours des modifications des données, assurer que la base obéit toujours aux contraintes.
        \item Lors de la création d'une contrainte, le SGBD vérifie qu'elle n'est pas en désaccord avec les valeurs présentes dans la base.
              \begin{itemize}
                  \item En désaccord : rejetée
                  \item Pas en désaccord : enregistrée et immédiatement opérationnelle
              \end{itemize}
    \end{itemize}
\end{frame}

\begin{frame}{\secname}
    \begin{itemize}
        \item SQL 2 fait la distinction entre
              \begin{itemize}
                  \item contraintes générales qui peuvent faire intervenir plusieurs colonnes de plusieurs tables
                  \item contraintes attachées aux tables de base
              \end{itemize}
        \item Remarque : une contrainte attachée à une table ne peut exister sans la table et est donc effacée en même temps que la table
    \end{itemize}
\end{frame}
\section{Contraintes attachées aux tables}
\begin{frame}{\secname}
    Les contraintes attachées à une table se définissent lors de la création de la table (CREATE TABLE) ou après la création de la table par la commande ALTER TABLE.
    Voir le chapitre 3 : LDD
\end{frame}

\section{Particularités d'Oracle}
\subsection{DEFERRABLE}
\begin{frame}{\secname : \subsecname}
    Oracle possède la clause \lstinline[language=plsql]![NOT] DEFERRABLE!
    Pour Oracle, une contrainte déclarée \lstinline[language=plsql]!DEFERRABLE! pourra être différée pendant la durée d'une transaction à condition de le préciser explicitement au moyen de la commande \lstinline[language=plsql]!SET CONSTRAINT!
\end{frame}

\begin{frame}{\secname : \subsecname}
    \lstinputlisting[language=plsql, title=Exemple]{../exemples/Chapitre 8/DEFERRABLE.sql}
    \textbf{La contrainte N'EST PAS différée !}
    Pour qu'elle le soit : \\
    \lstinline[language=plsql]!SET CONSTRAINT c DEFERRED;!
\end{frame}
\subsection{IMMEDIATE}

\begin{frame}{\secname : \subsecname}
    \lstinputlisting[language=plsql, title=Exemple]{../exemples/Chapitre 8/DEFERRABLE2.sql}
    \textbf{Pour inverser le comportement d'une contraint INITIALLY DEFERRED, on utilise la commande}
    Pour qu'elle le soit : \\
    \lstinline[language=plsql]!SET CONSTRAINT c IMMEDIATE;!
\end{frame}

\subsection{En résumé}
\begin{frame}{\secname : \subsecname}
    \begin{itemize}
        \item \lstinline[language=plsql]!INITIALLY IMMEDIATE! correspond à \lstinline[language=plsql]!NOT DEFERRABLE! de la norme et est le comportement par défaut
        \item \lstinline[language=plsql]!INITIALLY DEFERRED! est équivalent à \lstinline[language=plsql]!DEFERRABLE! de la norme
    \end{itemize}
\end{frame}
\section{Les déclencheurs}
\subsection{Introduction}
\begin{frame}{\secname : \subsecname}
    \begin{itemize}
        \item Un déclencheur (trigger) permet de définir un ensemble d'actions qui sont déclenchées automatiquement par le SGBD lorsque certains phénomènes se produisent.
        \item Les actions sont enregistrées dans la base et non plus dans les programmes d'application.
        \item Cette notion n'est pas spécifiée dans SQL2, elle le sera dans SQL3.
    \end{itemize}
\end{frame}

\begin{frame}{\secname : \subsecname}
    Les déclencheurs permettent de définir des contraintes dynamiques, ils peuvent être utilisés pour :
    \begin{itemize}
        \item Générer automatiquement une valeur de clé primaire,
        \item Résoudre le problème des mises à jour en cascade
        \item Enregistrer les accès à une table
        \item Gérer automatiquement la redondance
        \item Empêcher la modification par des personnes non autorisées (dans le domaine de la confidentialité)
        \item Mettre en œuvre des règles de fonctionnement plus complexes
    \end{itemize}
\end{frame}
\subsection{Déclencheurs dans Oracle}
\begin{frame}{\secname : \subsecname}
    \begin{itemize}
        \item Les déclencheurs permettent de réaliser des opérations sophistiquées, car ils constituent un bloc \textbf{PL/SQL}
        \item Depuis Oracle8i, il est possible de définir des déclencheurs s'activant suite à des \textbf{commandes du LDD} ou à certains événements systèmes
        \item Nous nous concentrerons sur les déclencheurs réagissant à l'exécution de commandes du LMD
        \item Il existe 12 types de déclencheurs sensibles aux \textbf{commandes LMD} en fonction de l'instruction déclenchante (ajout, suppression, modification), du niveau du déclencheur (ligne ou table) et du moment du déclenchement (avant ou après)
    \end{itemize}
\end{frame}
\subsection{Syntaxe}
\begin{frame}{\secname : \subsecname}
    \lstinputlisting[language=bnf, title=Syntaxe]{../exemples/Chapitre 8/create-trigger.bnf}
\end{frame}
\subsection{FOR EACH ROW}
\begin{frame}{\secname : \subsecname}
    \lstinline[language=plsql]!FOR EACH ROW! : déclencheur du niveau tuple : sera exécuté pour toutes les lignes provoquant l'activation du déclencheur.  En son absence, le déclencheur est du niveau table et le bloc PL/SQL n'est exécuté qu'une seule fois.
\end{frame}

\subsection{WHEN}
\begin{frame}{\secname : \subsecname}
    \lstinline[language=plsql]!WHEN! : permet de préciser une condition supplémentaire. Le bloc PL/SQL ne sera exécuté que si la condition de la clause WHEN est évaluée à vrai.
    Ex : vérifier, lors d'une mise à jour d'une colonne que la nouvelle valeur dépasse l'ancienne valeur.
\end{frame}

\subsection{restrictions}
\begin{frame}{\secname : \subsecname}
    Deux restrictions sur les commandes du bloc PL/SQL d'un déclencheur :
    \begin{itemize}
        \item Un déclencheur ne peut contenir de \lstinline[language=plsql]!COMMIT! ni de \lstinline[language=plsql]!ROLLBACK!
        \item Il est impossible d'exécuter une commande du LDD dans un déclencheur
        \item Un déclencheur  de niveau de ligne ne peut pas :
              \begin{itemize}
                  \item Lire ou modifier le contenu d'une table mutante
                  \item Lire ou modifier les colonnes d'une clé primaire, unique ou étrangère d'une table contraignante!
              \end{itemize}
    \end{itemize}
\end{frame}

\subsection{Table Mutante}
\begin{frame}{\secname : \subsecname}
    Table Mutante : table en cours de modification. Pour un déclencheur, il s'agit de la table sur laquelle il est défini
\end{frame}

\subsection{Table contraignante}
\begin{frame}{\secname : \subsecname}
    Table contraignante : table qui peut éventuellement être accédée en lecture afin de vérifier une contrainte de référence
\end{frame}
\subsection{Exemple}
\begin{frame}{\secname : \subsecname}
    \lstinputlisting[language=plsql, title=Exemple : contrainte dynamique]{../exemples/Chapitre 8/contrainte-dynamique.sql}
    Exemple qui ne fait que lancer une exception
    \lstinline[language=plsql]!Raise\_application\_error! : permet de transmettre un code d'erreur et un message au bloc appelant
    \lstinline[language=plsql]!OLD! et \lstinline[language=plsql]!NEW! sont prédéfinis, accèdent aux anciennes et nouvelles valeurs
\end{frame}

\begin{frame}[allowframebreaks]{\secname : \subsecname}
    Dans la table Service : colonne \lstinline[language=plsql]!Nombre\_Emp! contient le nombre d'employés de chaque service $\implies$ redondance
    \lstinputlisting[language=plsql, title=Exemple: gestion automatique de la redondance]{../exemples/Chapitre 8/redondance.sql}
    Pour assurer la cohérence des données, on crée la table Service avec 0 comme valeur par défaut pour la colonne \lstinline[language=plsql]!Nombre\_Emp! et on gère la redondance au moyen d'un déclencheur.
\end{frame}

\begin{frame}{\secname : \subsecname}
    \lstinputlisting[language=plsql, title=Exemple: gestion automatique de la redondance]{../exemples/Chapitre 8/redondance2.sql}
\end{frame}


\subsection{Mise à jour en cascade}
\begin{frame}{\secname : \subsecname}
    Il s'agit d'assurer qu'une mise à jour s'effectue uniformément dans plusieurs tables.
    Ce problème se pose
    \begin{itemize}
        \item  Lorsqu'une clé primaire est modifiée
        \item Lorsqu'on supprime d'une table un tuple qui est référencé par d'autres tuples dans d'autres tables
    \end{itemize}
\end{frame}

\begin{frame}{\secname : \subsecname}
    \lstinputlisting[language=plsql, title=Exemple : mise à jour en cascade]{../exemples/Chapitre 8/redondance3.sql}
\end{frame}

\end{document}