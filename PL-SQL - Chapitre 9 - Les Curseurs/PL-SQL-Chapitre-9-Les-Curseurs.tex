\documentclass[10pt]{beamer}

\input{/Users/daniel/Documents/LaTeX/beamer-style.tex}

\title{SGBD - 2\textsuperscript{e}}
\subtitle{PL-SQL - Chapitre 9 - Les Curseurs}
\date{\today}
\author{Anne Léonard}
\institute{Haute École de la Province de Liège}
%\titlegraphic{\hfill\includegraphics[height=1.5cm]{logo.eps}}

\begin{document}
\maketitle

\setbeamerfont{subsection in toc}{size=\small}
\setbeamerfont{subsubsection in toc}{size=\normalsize}
\setbeamertemplate{section in toc}[sections numbered]
\setbeamertemplate{subsection in toc}[subsections numbered]
\setbeamertemplate{subsubsection in toc}[subsubsections numbered]
\begin{frame}[allowframebreaks]{Table des matières du chapitre}
    \tableofcontents[subsectionstyle=show/show/hide,subsubsectionstyle=show/show/hide,]
\end{frame}

\section{Introduction}
\begin{frame}{\secname}
    \begin{itemize}
        \item Les curseurs permettent de traiter les requêtes de recherche dont le résultat contient un nombre quelconque de tuples.
        \item La notion de curseur réalise le lien entre la philosophie ensembliste de SQL (set-retrieval) et celle, enregistrement par enregistrement (record-at-a-time) des langages de 3ème génération.
    \end{itemize}
\end{frame}

\begin{frame}{\secname}
    \begin{itemize}
        \item Un curseur peut être considéré comme une espèce de pointeur\footnote{Il peut être utilisé pour parcourir un ensemble ordonné de tuples (active set};
        \item Un curseur est toujours associé à une expression de sélection.
    \end{itemize}
\end{frame}

\begin{frame}{\secname}
    \lstinputlisting[language=plsql, title=Exemple simple]{../exemples/PLSQL Chapitre 9/exemple1.sql}
\end{frame}

\begin{frame}{\secname}
    \begin{itemize}
        \item L'expression de sélection n'est pas évaluée au moment de la déclaration
        \item L'expression de sélection est évaluée lorsque le curseur est ouvert
        \item À ce moment, l'ensemble des tuples du résultat de l'expression de sélection est associé au curseur.
        \item L'association se maintient jusqu'à la fermeture du curseur
        \item Le résultat de l'expression de sélection est figé jusqu'à la fermeture du curseur
    \end{itemize}
\end{frame}


\section{Curseurs explicites}
\begin{frame}{\secname : \subsecname}
    \begin{itemize}
        \item Un curseur peut être déclaré dans un bloc PL/SQL, une procédure ou un package
              \lstinline[language=plsql]! CURSOR nom_curseur IS expression_de_sélection;!
        \item Aucune restriction au niveau de la requête de sélection
        \item Un curseur DOIT ÊTRE déclaré avant d'être utilisé.  Sinon : exception \lstinline[language=plsql]!invalid_cursor!
        \item 3 opérations pour manipuler les curseurs : \lstinline[language=plsql]!open!, \lstinline[language=plsql]!fetch! et \lstinline[language=plsql]!close!
    \end{itemize}
\end{frame}

\subsection{Open}
\begin{frame}{\secname : \subsecname}
    \begin{itemize}
        \item \lstinline[language=plsql]!OPEN nom_curseur!;
        \item L'effet du \lstinline[language=plsql]!OPEN! est d'évaluer l'expression de sélection associée.
        \item Le curseur est dans l'état ouvert et est positionné avant la première ligne du résultat
        \item Un curseur ouvert peut être parcouru par une série de \lstinline[language=plsql]!fetch!
        \item La tentative d'ouverture d'un curseur déjà ouvert provoque l'exception \lstinline[language=plsql]!cursor_already_open!
    \end{itemize}
\end{frame}

\subsection{FETCH}
\begin{frame}{\secname : \subsecname}
    \lstinputlisting[language=plsql, title=Récupérer les données]{../exemples/PLSQL Chapitre 9/exemple2.sql}
\end{frame}


\begin{frame}{\secname : \subsecname}
    \begin{itemize}
        \item Les variables de la liste ou les champs du record doivent être compatibles avec la clause select
        \item Idem si on utilise \lstinline[language=plsql]!BULK COLLECT INTO!
    \end{itemize}
\end{frame}

\subsection{CLOSE}
\begin{frame}{\secname : \subsecname}
    \begin{itemize}
        \item \lstinline[language=plsql]!CLOSE nom_curseur!;
        \item La fermeture d'un curseur non ouvert déclenche l'exception \lstinline[language=plsql]!invalid_cursor!
    \end{itemize}
\end{frame}


\begin{frame}[allowframebreaks]{\secname}
    \lstinputlisting[language=plsql, title=Exemple complet]{../exemples/PLSQL Chapitre 9/exemple3.sql}
\end{frame}


\begin{frame}[allowframebreaks]{\secname}
    \lstinputlisting[language=plsql, title=Exemple complet 2]{../exemples/PLSQL Chapitre 9/exemple4.sql}
\end{frame}

\begin{frame}{\secname}
    Chaque curseur possède 4 attributs :
    \begin{itemize}
        \item \lstinline[language=plsql]!\%FOUND! vaut \lstinline[language=plsql]!TRUE! si un \lstinline[language=plsql]!FETCH! a pu extraire un tuple
        \item \lstinline[language=plsql]!\%NOTFOUND! vaut \lstinline[language=plsql]!TRUE! si un \lstinline[language=plsql]!FETCH! n'a pas pu extraire de tuple
        \item \lstinline[language=plsql]!\%ISOPEN! vaut \lstinline[language=plsql]!TRUE! si le curseur est ouvert
        \item \lstinline[language=plsql]!\%ROWCOUNT! vaut le nombre de tuples qui ont déjà été extraits par des \lstinline[language=plsql]!FETCH!
    \end{itemize}
\end{frame}

\begin{frame}{\secname}
    \begin{table}[]
        \resizebox{\textwidth}{!}{%
            \begin{tabular}{|l|l|l|l|l|l|}
                \hline
                                                          &       & \lstinline[language=plsql]!\%FOUND! & \lstinline[language=plsql]!\%NOTFOUND! & \lstinline[language=plsql]!\%ISOPEN! & \lstinline[language=plsql]!\%ROWCOUNT! \\ \hline
                \lstinline[language=plsql]!OPEN!          & Avant & Exception                           & Exception                              & \lstinline[language=plsql]!FALSE!    & Exception                              \\ \hline
                                                          & Après & \lstinline[language=plsql]!NULL!    & \lstinline[language=plsql]!NULL!       & \lstinline[language=plsql]!TRUE!     & \lstinline[language=plsql]!0!          \\ \hline
                1er \lstinline[language=plsql]!FETCH!     & Avant & \lstinline[language=plsql]!NULL!    & \lstinline[language=plsql]!NULL!       & \lstinline[language=plsql]!TRUE!     & \lstinline[language=plsql]!0!          \\ \hline
                                                          & Après & \lstinline[language=plsql]!TRUE!    & \lstinline[language=plsql]!FALSE!      & \lstinline[language=plsql]!TRUE!     & 1                                      \\ \hline
                Nème \lstinline[language=plsql]!FETCH!    & Avant & \lstinline[language=plsql]!TRUE!    & \lstinline[language=plsql]!FALSE!      & \lstinline[language=plsql]!TRUE!     & \lstinline[language=plsql]!*!          \\ \hline
                                                          & Après & \lstinline[language=plsql]!TRUE!    & \lstinline[language=plsql]!FALSE!      & \lstinline[language=plsql]!TRUE!     & \lstinline[language=plsql]!*!          \\ \hline
                Dernier \lstinline[language=plsql]!FETCH! & Avant & \lstinline[language=plsql]!TRUE!    & \lstinline[language=plsql]!FALSE!      & \lstinline[language=plsql]!TRUE!     & \lstinline[language=plsql]!*!          \\ \hline
                                                          & Après & \lstinline[language=plsql]!FALSE!   & \lstinline[language=plsql]!TRUE!       & \lstinline[language=plsql]!TRUE!     & \lstinline[language=plsql]!*!          \\ \hline
                \lstinline[language=plsql]!CLOSE!         & Avant & \lstinline[language=plsql]!FALSE!   & \lstinline[language=plsql]!TRUE!       & \lstinline[language=plsql]!TRUE!     & \lstinline[language=plsql]!*!          \\ \hline
                                                          & Après & Exception                           & Exception                              & \lstinline[language=plsql]!FALSE!    & Exception                              \\ \hline
            \end{tabular}
        }
    \end{table}
\end{frame}

\begin{frame}{\secname}
    Légende du tableau :
    \begin{itemize}
        \item \lstinline[language=plsql]!*! signifie que le résultat dépend du nombre de \lstinline[language=plsql]!FETCH! exécutés
        \item Après le 1er \lstinline[language=plsql]!FETCH!, si le résultat est vide, \lstinline[language=plsql]!\%FOUND! donne \lstinline[language=plsql]!FALSE!, \lstinline[language=plsql]!\%NOTFOUND! donne \lstinline[language=plsql]!TRUE! et \lstinline[language=plsql]!\%ROWCOUNT! donne \lstinline[language=plsql]!0!
        \item \lstinline[language=plsql]!\%ISOPEN! est souvent utilisé dans le traitement des exceptions pour fermer le ou les curseurs ouverts
    \end{itemize}
    \lstinputlisting[language=plsql]{../exemples/PLSQL Chapitre 9/exemple5.sql}
\end{frame}


\section{Curseurs for implicites}
\begin{frame}{\secname}
    Du point de vue programmation : ce sont les plus simples ! Il suffit de placer la requête de sélection.
    \lstinputlisting[language=plsql]{../exemples/PLSQL Chapitre 9/exemple6.sql}
\end{frame}


\section{Curseurs for explicites}
\begin{frame}{\secname}
    \begin{itemize}
        \item Le curseur est déclaré de manière explicite;
        \item Son parcours est réalisé au moyen d'une boucle \lstinline[language=plsql]!for! et d'un record ad hoc implicitement déclaré dans la boucle;
        \item Il est possible d'utiliser les attributs \lstinline[language=plsql]!\%FOUND!, \lstinline[language=plsql]!\%NOTFOUND!, \lstinline[language=plsql]!\%ISOPEN!, \lstinline[language=plsql]!\%ROWCOUNT!.
    \end{itemize}
\end{frame}

\begin{frame}{\secname}
    \lstinputlisting[language=plsql, title=Curseurs for explicites]{../exemples/PLSQL Chapitre 9/exemple7.sql}
\end{frame}



\section{Curseurs avec paramètres}
\begin{frame}{\secname}
    \begin{itemize}
        \item Il est possible de passer des paramètres à un curseur
        \item La déclaration des paramètres dans le curseur se fait lors de la déclaration du curseur, de la même manière que pour une procédure
        \item L'évaluation et la transmission des paramètres se fait au moment de l'ouverture du curseur
    \end{itemize}
\end{frame}

\begin{frame}{\secname}
    \lstinputlisting[language=plsql, title=Curseurs avec paramètres]{../exemples/PLSQL Chapitre 9/exemple8.sql}
\end{frame}

\begin{frame}[allowframebreaks]{\secname}
    \lstinputlisting[language=plsql, title=Curseurs avec paramètres]{../exemples/PLSQL Chapitre 9/exemple9.sql}
\end{frame}

\section{Curseurs for update}
\begin{frame}{\secname}
    \begin{itemize}
        \item Les curseurs for update permettent de modifier le tuple qui vient d'être extrait par un \lstinline[language=plsql]!FETCH!
        \item Oracle place des verrous exclusifs sur les tuples de l'active set des curseurs for update
    \end{itemize}
    \lstinputlisting[language=plsql]{../exemples/PLSQL Chapitre 9/exemple10.sql}
\end{frame}
% \section{Commit et les curseurs}
% \begin{frame}[allowframebreaks]{\secname}
%     \lstinputlisting[language=plsql, title=Modifier le salaire de certains employés]{../exemples/PLSQL Chapitre 9/exemple11.sql}
% \end{frame}

% \begin{frame}[allowframebreaks]{\secname}
%     \begin{itemize}
%         \item Le \lstinline[language=plsql]!COMMIT! DOIT être placé en dehors de la boucle car il ferme les curseurs ouverts
%         \item Dès lors placer le \lstinline[language=plsql]!COMMIT! dans la boucle provoque l'erreur "ORA-1002 : fetch out of sequence" lors du \lstinline[language=plsql]!FETCH! suivant
%         \item Inconvénient : l'active set  peut être important et des tuples peuvent rester verrouillés longtemps après que la transaction les ait modifiés
%     \end{itemize}
% \end{frame}

\section{Le curseur implicite SQL}
\begin{frame}{\secname}
    \begin{itemize}
        \item PL/SQL utilise un curseur implicite pour chaque opération du LMD de SQL (\lstinline[language=plsql]!INSERT!, \lstinline[language=plsql]!UPDATE!, \lstinline[language=plsql]!DELETE!);
        \item Ce curseur porte le nom SQL et il est exploitable après avoir exécuté l'instruction;
    \end{itemize}
\end{frame}

\begin{frame}{\secname}
    Le curseur implicite SQL :
    \begin{itemize}
        \item Ne nécessite pas de déclaration;
        \item Ne doit pas être ouvert, ni fermé;
        \item Peut utiliser les attributs \lstinline[language=plsql]!\%FOUND!, \lstinline[language=plsql]!\%NOTFOUND!, \lstinline[language=plsql]!\%ISOPEN!, \lstinline[language=plsql]!\%ROWCOUNT!;
        \item Permet d'exécuter des \lstinline[language=plsql]!INSERT!, \lstinline[language=plsql]!UPDATE!, \lstinline[language=plsql]!DELETE!, \lstinline[language=plsql]!SELECT INTO! d'un seul tuple\footnote{Contrairement aux autres curseurs : recherche portant sur plusieurs tuples.}
    \end{itemize}
\end{frame}

\begin{frame}{\secname}
    \lstinputlisting[language=plsql, title=Mise à jour ou insertion]{../exemples/PLSQL Chapitre 9/exemple12.sql}
\end{frame}

\begin{frame}[allowframebreaks]{\secname}
    \lstinputlisting[language=plsql, title=Effacer le projet si ce projet a commencé il y plus de trois ans]{../exemples/PLSQL Chapitre 9/exemple13.sql}
\end{frame}
\input{/Users/daniel/Documents/LaTeX/bibliographie-anne.tex}

\end{document}
