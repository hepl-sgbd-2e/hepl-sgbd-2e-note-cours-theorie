\documentclass[10pt]{beamer}

\input{/Users/daniel/Documents/LaTeX/beamer-style.tex}

\title{SGBD - 2\textsuperscript{e}}
\subtitle{PL-SQL - Chapitre 7 - Les procédures et les fonctions}
\date{\today}
\author{Daniel Schreurs}
\institute{Haute École de Province de Liège}
%\titlegraphic{\hfill\includegraphics[height=1.5cm]{logo.eps}}

\begin{document}
\maketitle

\setbeamerfont{subsection in toc}{size=\small}
\setbeamerfont{subsubsection in toc}{size=\normalsize}
\setbeamertemplate{section in toc}[sections numbered]
\setbeamertemplate{subsection in toc}[subsections numbered]
\setbeamertemplate{subsubsection in toc}[subsubsections numbered]
\begin{frame}[allowframebreaks]{Table des matières du chapitre}
    \tableofcontents[subsectionstyle=show/show/hide,subsubsectionstyle=show/show/hide,]
\end{frame}

\section{CREATE PROCEDURE}
\begin{frame}{\secname}
    \lstinputlisting[language=plsql, title=Création d’une procédure]{../exemples/PLSQL Chapitre 7/exemple1.sql}
\end{frame}


\begin{frame}{\secname}
    Chaque procédure ou fonction peut comprendre des paramètres. Pour chaque paramètre, on doit spécifier :
    \begin{itemize}
        \item Son nom;
        \item Son mode d'accessibilité (\lstinline[language=plsql]!IN!, \lstinline[language=plsql]!OUT! ou \lstinline[language=plsql]!IN OUT!);
        \item Son type (pas de précision ni de longueur);
        \item Éventuellement sa valeur par défaut.
    \end{itemize}
\end{frame}

\begin{frame}{\secname}
    \lstinputlisting[language=plsql, title=Afficher l’employé]{../exemples/PLSQL Chapitre 7/exemple2.sql}
\end{frame}

\begin{frame}{\secname}
    \metroset{block=fill}
    \begin{alertblock}{Important}
        Une procédure ou une fonction peut également être déclarée localement dans une autre.
    \end{alertblock}
\end{frame}



\section{Compilation}
\begin{frame}{\secname}
    \lstinputlisting[language=plsql, title=Compilation sous SQLPLUS]{../exemples/PLSQL Chapitre 7/exemple3.sh}
    Le code est stocké dans le dictionnaire de données.\footnote{Les sources des objets (procédures, fonctions, packages) sont mémorisés dans la table \lstinline[language=plsql]!SOURCE\$! (propriétaire \lstinline[language=plsql]!SYS!).}
\end{frame}

\begin{frame}{\secname}
    Compilation sous SQLPLUS :
    \begin{itemize}
        \item Lors de la compilation d'un objet, le moteur PL/SQL génère les messages d'erreurs dans la table \lstinline[language=plsql]!ERROR$!
        \item Sous SQLPLUS, la commande SHOW ERRORS permet de visualiser les erreurs de compilation.
    \end{itemize}
\end{frame}

\section{Exécuter une procédure}
\begin{frame}{\secname}
    \lstinputlisting[language=plsql, title=Appeler une procédure]{../exemples/PLSQL Chapitre 7/exemple4.sql}
\end{frame}

\section{Création d’une fonction}
\begin{frame}{\secname}
    \lstinputlisting[language=plsql, title=Sortir d’une fonction]{../exemples/PLSQL Chapitre 7/exemple5.sql}
\end{frame}

\begin{frame}[allowframebreaks]{\secname}
    \lstinputlisting[language=plsql, title=Recherche]{../exemples/PLSQL Chapitre 7/exemple6.sql}
\end{frame}

\section{Exécuter une fonction}
\begin{frame}{\secname}
    \lstinputlisting[language=plsql, title=Recherche]{../exemples/PLSQL Chapitre 7/exemple7.sql}
\end{frame}

\section{Utiliser les paramètres avec \textit{NOCOPY}}
\begin{frame}{\secname}
    \begin{itemize}
        \item Par défaut, les paramètres \lstinline[language=plsql]!OUT! et \lstinline[language=plsql]!IN OUT! sont passés par valeur.\footnote{Les valeurs sont donc copiées avant l’exécution du  sous-programme.}
        \item Pendant l’exécution de celui-ci, des variables temporaires sont créées pour contenir les données des paramètres de type \lstinline[language=plsql]!OUT!.
        \item Si le sous-programme termine normalement son exécution, les valeurs sont alors recopiées dans les paramètres.
        \item \textbf{Ces copies ralentissent fortement les performances et encombrent la mémoire}.
        \item Pour éviter ces inconvénients : \lstinline[language=plsql]!NOCOPY!.
    \end{itemize}
\end{frame}

\section{\textit{RAISE\_APPLICATION\_ERROR}}
\begin{frame}{\secname}
    \lstinputlisting[language=plsql, title=Lancer une exception]{../exemples/PLSQL Chapitre 7/exemple8.sql}
    \begin{itemize}
        \item Définie dans \lstinline[language=plsql]!DBMS_STANDARD!
        \item Plage de codes d'erreur de -20000 à -20999
    \end{itemize}
\end{frame}

\begin{frame}[allowframebreaks]{\secname}
    \lstinputlisting[language=plsql, title=Fonction retournant un code d'erreur personnalisé]{../exemples/PLSQL Chapitre 7/exemple9.sql}
\end{frame}

\begin{frame}{\secname}
    \lstinputlisting[language=plsql, title=Appel de cette nouvelle fonction]{../exemples/PLSQL Chapitre 7/exemple10.sql}
\end{frame}

\section{Les notations nommées}

\begin{frame}{\secname}
    Les notations nommées sont nécessaires lors de l'appel :
    \begin{itemize}
        \item Pour utiliser les valeurs par défaut des paramètres
        \item Pour spécifier les paramètres dans n'importe quel ordre
    \end{itemize}
\end{frame}

\begin{frame}{\secname}
    \lstinputlisting[language=plsql, title=L’ordre des paramètres n’a plus d’importance]{../exemples/PLSQL Chapitre 7/exemple11.sql}
\end{frame}

\section{Les paramètres par défaut}

\begin{frame}{\secname}
    \lstinputlisting[language=plsql, title=Lorsque les paramètres sont spécifiés dans le mode IN il est possible de leur affecter des valeurs par défaut]{../exemples/PLSQL Chapitre 7/exemple12.sql}
\end{frame}


\end{document}
