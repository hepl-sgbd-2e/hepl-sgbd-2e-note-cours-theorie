\documentclass[10pt]{beamer}

\input{/Users/daniel/Documents/LaTeX/beamer-style.tex}

\title{Systèmes de Gestion de Bases de Données - 2e}
\subtitle{Chapitre 6 - Confidentialité des données}
\date{\today}
\author{Daniel Schreurs}
\institute{Haute École de la Province de Liège}
%\titlegraphic{\hfill\includegraphics[height=1.5cm]{logo.eps}}


\setbeamertemplate{frame footer}{\insertsectionhead}
\begin{document}
\maketitle

\setbeamerfont{subsection in toc}{size=\small}
\setbeamerfont{subsubsection in toc}{size=\normalsize}
\setbeamertemplate{section in toc}[sections numbered]
\setbeamertemplate{subsection in toc}[subsections numbered]
\setbeamertemplate{subsubsection in toc}[subsubsections numbered]
\begin{frame}[allowframebreaks]{Table des matières du chapitre}
    \tableofcontents[subsectionstyle=show/show/hide,subsubsectionstyle=show/show/hide,]
\end{frame}

\section{Introduction}
\tocss
\subsection{Définition}
\begin{frame}{\secname : \subsecname}
    La sécurité des données est un terme général contenant trois grands types de contrôle :
    \begin{itemize}
        \item Le contrôle d'accès au système par des utilisateurs non identifiés;
        \item Le contrôle de l'accès illégal aux données ou confidentialité;
        \item Le contrôle de la modification invalide des données ou intégrité.
    \end{itemize}
\end{frame}

\begin{frame}{\secname : \subsecname}
    Le contrôle d'accès au système par des utilisateurs non identifiés;\\
    On suppose que l'\textbf{identification de l'utilisateur} est prise en charge par le système d'exploitation.
\end{frame}

\begin{frame}{\secname : \subsecname}
    Le contrôle de la modification invalide des données ou intégrité.\\
    Le contrôle de l'intégrité est vaste. La gestion de la concurrence a été abordée au chapitre 5, l'intégrité sémantique le sera au chapitre 8 (contraintes d'intégrité et déclencheurs), la reprise après panne sort du cadre du cours.
\end{frame}

\begin{frame}{\secname : \subsecname}
    Le contrôle de l'accès illégal aux données ou confidentialité;\\
    Dans ce chapitre, nous allons donc nous attarder sur la confidentialité.
    Plus précisément, nous nous limiterons au contrôle des autorisations d'accès.
\end{frame}

\subsection{Que signifie "confidentialité" ?}
\begin{frame}{\secname : \subsecname}
    4 grandes classes de techniques propres à assurer la confidentialité d'un système manipulant des données :
    \begin{itemize}
        \item Contrôle du flux des données,
        \item Contrôle d'intégrité
        \item Encryptage des données
        \item Contrôle des autorisations d'accès
    \end{itemize}
\end{frame}

\begin{frame}{\secname : \subsecname}
    Les 3 premières (contrôle du flux des données, contrôle d'intégrité et chiffrement des données) sont impossibles à mettre en œuvre au moyen de SQL seul, ils font appel à des techniques particulières qui dépassent le cadre du cours.
    Nous étudierons en détail le contrôle des autorisations d'accès.
\end{frame}

\section{Mécanisme d'octroi et annulation de privilèges}
\subsection{Deux types}
\begin{frame}{\secname : \subsecname}
    Les privilèges sont de deux types :
    \begin{itemize}
        \item Les privilèges de niveau système
              \begin{itemize}
                  \item Permettent la création, modification, suppression de groupes d'objets.
                        Exemple : \lstinline[language=plsql]!CREATE TABLE!, \lstinline[language=plsql]!CREATE VIEW!, \lstinline[language=plsql]!CREATE SEQUENCE!, permettent, à l'utilisateur qui les a reçus de créer des tables, des vues et des séquences.
              \end{itemize}
              Les privilèges de niveau Objet
              \begin{itemize}
                  \item Permettent les manipulations sur des objets spécifiques.
                        Exemple : les privilèges \lstinline[language=plsql]!SELECT!, \lstinline[language=plsql]!INSERT!, \lstinline[language=plsql]!UPDATE!, \lstinline[language=plsql]!DELETE! sur la table \lstinline[language=plsql]!INFOSOFT.employes! permettent à l'utilisateur qui les a reçus de sélectionner, ajouter, modifier et supprimer des lignes dans la table EMPLOYES appartenant à l'utilisateur INFOSOFT.
              \end{itemize}
    \end{itemize}
\end{frame}

\subsection{Assigner des privilèges}
\begin{frame}{\secname : \subsecname}
    Assigner des privilèges système à un utilisateur
    \begin{itemize}
        \item Lorsqu'un utilisateur est créé avec l'instruction CREATE USER, il ne dispose d'aucun droit.  Il ne peut même pas se connecter à la base !
        \item Il doit pouvoir se connecter, créer des tables, des vues, des séquences.  Pour lui assigner ces privilèges de niveau système, il faut utiliser l'instruction \lstinline[language=plsql]!GRANT!
    \end{itemize}
\end{frame}

\begin{frame}{\secname : \subsecname}
    \lstinputlisting[language=plsql, title=Assigner des privilèges système à un utilisateur]{../exemples/Chapitre 6/grant-acces-1.sql}
\end{frame}

\begin{frame}{\secname : \subsecname}
    La liste des privilèges système assignés à l'utilisateur au cours de sa session est visible via la vue \lstinline[language=plsql]!SESSION\_PRIVS!
    Pour les voir, il suffit donc d'exécuter l'instruction :
    \lstinline[language=plsql]!SELECT * FROM SESSION\_PRIVS;!
\end{frame}
\begin{frame}{\secname : \subsecname}
    Au centre des mécanismes d'octroi/annulation et de contrôle des autorisations se trouve le dictionnaire des données (ou méta-base) dans lequel sont enregistrées toutes les autorisations d'accès.\\
    Pour SQL2, les privilèges sont enregistrés dans les tables
    \lstinline[language=plsql]!TABLE\_PRIVILEGES!, \lstinline[language=plsql]!COLUMN\_PRIVILEGE! et \lstinline[language=plsql]!USAGE\_PRIVILEGES! (\lstinline[language=plsql]!all\_tab\_privs! en Oracle)
\end{frame}

\begin{frame}{\secname : \subsecname}
    Le contrôle des autorisations (niveau objet)\\
    Le mécanisme d'octroi/annulation des privilèges permet à l'ADB ou à toute personne autorisée d'accorder ou de retirer les privilèges à des \lstinline[language=plsql]!sujets! sur des \textbf{objets}.
    \begin{itemize}
        \item \textbf{Sujets} : un utilisateur, un groupe d'utilisateur, tous les utilisateurs
        \item \textbf{Objets} : la base de données, les tables, les vues, les index, les procédures stockées, …
        \item \textbf{Privilèges} : varient d'un SGBD à l'autre, mais au minimum \lstinline[language=plsql]!SELECT!, \lstinline[language=plsql]!INSERT!, \lstinline[language=plsql]!DELETE!, \lstinline[language=plsql]!UPDATE!
    \end{itemize}
\end{frame}

\begin{frame}{\secname : \subsecname}
    Le mécanisme de contrôle d'autorisation est le mécanisme qui vérifie qu'un sujet donné a le droit d'effectuer une requête précise (lecture, mise à jour d'une table, création d'une table, …) sur un objet.  Ce contrôle est effectué en consultant les tables de la méta base.
\end{frame}

\begin{frame}{\secname : \subsecname}
    Principes généraux :
    \begin{itemize}
        \item Un utilisateur possède automatiquement tous les privilèges sur un objet qui lui appartient
        \item Un utilisateur ne peut pas donner plus de privilèges qu'il n'en a reçus
        \item S'il n'a pas reçu le privilège avec l'option \lstinline[language=plsql]!WITH GRANT OPTION!, un utilisateur ne peut pas donner à son tour ce privilège à un autre utilisateur
    \end{itemize}
\end{frame}

\begin{frame}{\secname : \subsecname}
    \lstinputlisting[language=plsql, title=Octroi de privilèges (niveau objet)]{../exemples/Chapitre 6/grant.bnf}
    \begin{itemize}
        \item \textbf{privilege} \lstinline[language=plsql]!SELECT!, \lstinline[language=plsql]!INSERT!, \lstinline[language=plsql]!INSERT(x)!, \lstinline[language=plsql]!UPDATE!, \lstinline[language=plsql]!UPDATE(x)!, \lstinline[language=plsql]!DELETE!, \lstinline[language=plsql]!ALL! (= tous les privilèges que le donneur peut accorder sur l'objet)
        \item \textbf{objet} liste de tables, vues ou colonnes précédée de TABLE ou VIEW suivant qu'il s'agit d'une table ou d'une vueutilisateur liste d'utilisateurs ou \lstinline[language=plsql]!PUBLIC!
        \item \lstinline[language=plsql]!WITH GRANT OPTION! : permet au donneur d'indiquer que le receveur pourra transmettre les privilèges qu'il reçoit.
    \end{itemize}

\end{frame}


\begin{frame}{\secname : \subsecname}
    \lstinputlisting[language=plsql, title=Exemple : Le directeur possède tous les privilèges sur la table RESULTATS et peut les transmettre.]{../exemples/Chapitre 6/grant-acces-2.sql}
\end{frame}

\begin{frame}{\secname : \subsecname}
    \lstinputlisting[language=bnf, title=Privilège objet et utilisateur ont la même signification que celle de la commande GRANT]{../exemples/Chapitre 6/remove-acces-1.bnf}
    \lstinline[language=plsql]!REVOKE SELECT ON TABLE t FROM Obelix;!
\end{frame}

\section{Complément SQL2}
\begin{frame}{\secname : \subsecname}
    Pour plus d'information, voir le livre de référence.
\end{frame}

\section{Les possibilités d'Oracle}
\begin{frame}{\secname : \subsecname}
    Pour plus d'information, voir le livre de référence.
\end{frame}

\begin{frame}{\secname : \subsecname}
    Pour plus d'informations, consulter le site :
    \href{http://oracle.developpez.com/guide/administration/adminrole/}{oracle.developpez.com}
\end{frame}

\end{document}