\documentclass[10pt]{beamer}

\input{/Users/daniel/Documents/LaTeX/beamer-style.tex}


\title{SGBD - 2\textsuperscript{e}}
\subtitle{Présentation}
\date{\today}
\author{Daniel Schreurs}
\institute{Haute École de la Province de Liège}
%\titlegraphic{\hfill\includegraphics[height=1.5cm]{logo.eps}}

\begin{document}

\maketitle

\setbeamerfont{subsection in toc}{size=\small}
\begin{frame}[allowframebreaks]{Table des matières}
    \setbeamertemplate{section in toc}[sections numbered]
    \tableofcontents
\end{frame}

\section{Introduction}

\subsection{Informations relatives au cours}
\begin{frame}{\secname : \subsecname}
    \begin{itemize}
        \item Mon adresse mail : \href{mailto:daniel.schreurs@hepl.be}{daniel.schreurs@hepl.be}
        \item Moodle : \href{https://moodle.ecolevirtuelle.be/course/view.php?id=25655}{Système de gestion de bases de données - Théorie}
        \item Forum du cours : \href{https://moodle.ecolevirtuelle.be/mod/forum/view.php?id=183803}{Poser une question}
    \end{itemize}
\end{frame}

\subsection{Répartition du cours}
\begin{frame}{\secname : \subsecname}
    \begin{table}[]
        \begin{tabular}{ll}
            Théorie 15 heures   & Laboratoire 15 heures \\
            Examen écrit : 50\% & Examen oral : 50\%
        \end{tabular}
        \caption*{Bachelier en informatique et systèmes}
    \end{table}
\end{frame}

\subsection{Les objectifs de ce cours}
\begin{frame}{\secname : \subsecname}
    \begin{itemize}
        \item Décrire et utiliser les fonctions d’un système de gestion de bases données.
        \item Décrire le fonctionnement interne d’un système de gestion de bases de données et ses interactions avec le système d’exploitation
        \item Interpréter du code écrit dans un langage relationnel de type SQL sous ses aspects de description, manipulation et contrôle des données.
        \item Comprendre et employer le modèle relationnel de données
        \item Employer des instructions de création d'une base de données à partir d’un modèle logique
        \item Utiliser le SQL afin de manipuler des données (ajout, recherche, suppression)
    \end{itemize}
\end{frame}
\subsection{Aperçu du contenu du cours de théorie}
\begin{frame}{\secname : \subsecname}
    \begin{itemize}
        \item Chapitre 1 : Concepts de base
        \item Chapitre 2 : Modèle relationnel
        \item Chapitre 3 : Langage de définition des données - LDD
        \item Chapitre 4 : Langage de manipulation des données - LMD
        \item Chapitre 5 : Transactions et accès concurrents - LCD
        \item Chapitre 6 : Confidentialité des données
        \item Chapitre 7 : Vues
        \item Chapitre 8 : Contraintes d'intégrité et déclencheurs
        \item Chapitre 9 : PL-SQL
    \end{itemize}
\end{frame}

\section{Sondage}
\subsection{Langage de manipulation de données}
\begin{frame}{\secname : \subsecname}
    \begin{itemize}
        \item Valeur NULL
        \item Opérateur CASE
        \item Jointure (produit cartésien, naturelle, externe)
        \item Opérateurs ensemblistes (union, différence, intersection)
        \item Requêtes imbriquées (IN, ALL, ANY, SOME, EXISTS)
        \item UPDATE, DELETE, INSERT
    \end{itemize}
\end{frame}

\subsection{Langage de définition de données}
\begin{frame}{\secname : \subsecname}
    \begin{itemize}
        \item Notion de schéma et d’utilisateurs
        \item CREATE TABLE
        \item Notion de clé (primaire, étrangère)
        \item Contraintes d’intégrité (domaine, entité, relation, référence)
        \item Notion d’index (et cluster)
        \item ALTER TABLE
        \item DROP TABLE
    \end{itemize}
\end{frame}

\subsection{Transactions}
\begin{frame}{\secname : \subsecname}
    \begin{itemize}
        \item Transactions concurrentes et cohérences
        \item Transactions ACID
        \item Niveaux d’isolation
        \item COMMIT, ROLLBACK?
        \item Verrous explicites (LOCK, SELECT FOR UPDATE)?
        \item Verrouillage en deux phases (two-phase locking)?
    \end{itemize}
\end{frame}

\subsection{Divers}
\begin{frame}{\secname : \subsecname}
    \begin{itemize}
        \item Gestion des droits d’accès des utilisateurs?
        \item Les vues et les tables temporaires
        \item Les déclencheurs (triggers)?
    \end{itemize}
\end{frame}



\end{document}
