\documentclass[10pt]{beamer}

\input{/Users/daniel/Documents/LaTeX/beamer-style.tex}

\title{SGBD - 2\textsuperscript{e}}
\subtitle{PL-SQL - Chapitre 8 - Les packages}
\date{\today}
\author{Daniel Schreurs}
\institute{Haute École de Province de Liège}
%\titlegraphic{\hfill\includegraphics[height=1.5cm]{logo.eps}}

\begin{document}
\maketitle

\setbeamerfont{subsection in toc}{size=\small}
\setbeamerfont{subsubsection in toc}{size=\normalsize}
\setbeamertemplate{section in toc}[sections numbered]
\setbeamertemplate{subsection in toc}[subsections numbered]
\setbeamertemplate{subsubsection in toc}[subsubsections numbered]
\begin{frame}[allowframebreaks]{Table des matières du chapitre}
    \tableofcontents[subsectionstyle=show/show/hide,subsubsectionstyle=show/show/hide,]
\end{frame}

\section{Le concept du package}

\begin{frame}{\secname}
    \begin{itemize}
        \item Un package est composé de \textbf{deux parties} de code bien distinctes : la \textbf{spécification} (specification) et le \textbf{corps} (body);
        \item La \textbf{spécification} du package contient les prototypes des procédures et fonctions et la déclaration des variables publiques
        \item Le \textbf{corps} du package contient l'implémentation des éléments définis dans la spécification.  Le corps peut également contenir et définir des éléments privés.
        \item Le corps du package peut également contenir une section d'instructions appelée initialization section.
    \end{itemize}
\end{frame}

\begin{frame}{\secname}
    \begin{itemize}
        \item Le code défini publiquement est accessible à tout utilisateur possédant le \textbf{privilège} \lstinline[language=plsql]!EXECUTE! sur le package
        \item La sécurité repose sur des \lstinline[language=plsql]!GRANT EXECUTE!. Aucun accès aux données n'est accordé, uniquement aux procédures, fonctions et packages.
        \item Le code privé est inaccessible : ceci renforce encore l'indépendance données-programmes et la réutilisabilité
    \end{itemize}
\end{frame}

\begin{frame}{\secname}
    \lstinputlisting[language=plsql, title=Le serveur de bases de données assure que le code n'est exécuté qu'une seule fois par session lorsque le package est instancié]{../exemples/PLSQL Chapitre 8/exemple1.sql}
\end{frame}

\section{Persistance de la session}
\begin{frame}{\secname}
    Il est possible :
    \begin{itemize}
        \item D'établir une session;
        \item D'exécuter un programme qui assigne une valeur à une variable d'un package;
    \end{itemize}
    Cette variable est donc initialisée et persistera jusqu'à la fin de la session
\end{frame}

\begin{frame}{\secname}
    L'exemple le plus simple est celui d'un package ne comprenant qu'une spécification où une \textbf{variable publique} est déclarée.
    \lstinputlisting[language=plsql]{../exemples/PLSQL Chapitre 8/exemple3.sql}
\end{frame}

\begin{frame}{\secname}
    Accès à la variable publique :
    \begin{itemize}
        \item Tout utilisateur ayant reçu le privilège d'exécuter le package possède une instance de celui-ci dès qu'il y accède pour la première fois
        \item Ainsi, un tel utilisateur pourra exécuter la portion de code suivante où INPRES est le nom du schéma propriétaire du package.
    \end{itemize}
    \lstinputlisting[language=plsql, title=Accès à la variable publique]{../exemples/PLSQL Chapitre 8/exemple4.sql}
\end{frame}

\begin{frame}{\secname}
    Pour rendre une variable privée, il suffit de ne pas la mettre dans la spécification.
    \lstinputlisting[language=plsql]{../exemples/PLSQL Chapitre 8/exemple5.sql}
\end{frame}

\begin{frame}{\secname}
    \metroset{block=fill}
    \begin{alertblock}{Important}
        Les variables publiques sont locales à une session.
    \end{alertblock}
\end{frame}

\section{Principaux avantages des packages}
\begin{frame}{\secname}
    \begin{itemize}
        \item Un package permet de regrouper plusieurs unités de programmation dans un même container réduction du nombre d'objets stockés à maintenir
        \item Le fait de séparer l'implémentation de l'interface permet de maintenir ou modifier l'implémentation sans toucher à l'interface ni aux programmes appelants.
        \item Il n'est même pas nécessaire de recompiler les programmes appelants si l'interface n'est pas elle-même recompilée
    \end{itemize}
\end{frame}

\section{Procédures et fonctions polymorphes et surchargées}
\begin{frame}{\secname}
    \lstinputlisting[language=plsql, title=Spécification du package contenant 2 fonctions polymorphes]{../exemples/PLSQL Chapitre 8/exemple11.sql}
\end{frame}

\begin{frame}[allowframebreaks]{\secname}
    \lstinputlisting[language=plsql, title=Body du package contenant 2 fonctions polymorphes]{../exemples/PLSQL Chapitre 8/exemple12.sql}
\end{frame}

\end{document}
